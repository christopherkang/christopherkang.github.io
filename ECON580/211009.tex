\documentclass{article}
\usepackage[dvipsnames]{xcolor}
\usepackage[utf8]{inputenc}
\usepackage{amsmath, amsfonts}
\usepackage{braket}
\usepackage{graphicx}
\usepackage{mathtools}
% \usepackage{thmbox}
\usepackage{amsthm}
\usepackage{qcircuit}
\usepackage{dsfont}
\usepackage{tikz}
\usepackage{mathtools}
\usepackage{enumerate}

\usepackage{caption}
\usepackage[colorlinks,allcolors=blue]{hyperref} % optional
\usepackage[noabbrev,capitalize,nameinlink]{cleveref}
\usepackage[ruled,vlined]{algorithm2e}
\usepackage[style=numeric, sorting=none]{biblatex}
\addbibresource{references.bib}

\usepackage{geometry}
\geometry{margin=1in}

\title{ECON Brainstorming}
\author{Christopher Kang \\
\href{https://www.overleaf.com/read/xbkhntchwshn}{Click for Latest Version}}
\date{\today}

\newcommand{\kron}{\otimes}
\newcommand{\norm}[1]{\left\lVert#1\right\rVert}
\newcommand{\identity}{\mathds{1}}
\newcommand{\how}[1]{\colorbox{BurntOrange}{\textbf{#1}}}
\DeclareMathOperator{\Tr}{Tr}

\DeclarePairedDelimiter\ceil{\lceil}{\rceil}
\DeclarePairedDelimiter\floor{\lfloor}{\rfloor}

\newtheorem{theorem}{Theorem}[section]
\newtheorem{claim}{Claim}[section]
\newtheorem{lemma}{Lemma}[section]
\newtheorem{fact}{Fact}
\newtheorem{corr}{Corollary}[section]
\newtheorem{define}{Definition}
\newtheorem{problem}{Problem}[section]
\newtheorem{practice}{Practice}[section]
\newtheorem{question}{Question}

\newcommand{\indep}{\perp \!\!\! \perp}

\renewcommand{\thesubsection}{\thesection.\alph{subsection}}

\begin{document}

\maketitle

We will begin by restating some useful theorems:
\how{fill me in}

%%1
\section{}
\subsection{}
Yes, $A$ and $B$ can be mutually exclusive. Recall that:
\begin{align}
    \mathbb{P}[A \cup B] = \mathbb{P}[A] + \mathbb{P}[B] - \mathbb{P}[A \cap B] \implies \mathbb{P}[A \cap B] = \mathbb{P}[A] + \mathbb{P}[B] - \mathbb{P}[A \cup B]
\end{align}

We also know that: \how{need to resolve this still}
\begin{align}
    \max(\mathbb{P}[A], \mathbb{P}[B]) \leq \mathbb{P}[A \cup B] \leq
\end{align}

\how{I need to explain this lol}

\subsection{}

\subsection{}

\subsection{}

\subsection{}
No. Suppose, by contradiction, that $A, B$ mutually exclusive and $A \indep B$. If this were the case, then:
\begin{align}
    \mathbb{P}[A \cap B] = 0 = \mathbb{P}[A] \mathbb{P}[B]
\end{align}

However, we know that $\mathbb{P}[A] \mathbb{P}[B] = 0.12$, a contradiction. Thus, $A, B$ cannot be both. 

\subsection{}
Recall our equation from (e):
\begin{align}
    \mathbb{P}[A \cap B] = 0 = \mathbb{P}[A] \mathbb{P}[B]
\end{align}

This is true if and only if one of the probabilities take on the value $0$. Thus, because all probabilities are nonnegative,
\begin{align}
    \min \{ \mathbb{P}[C], \mathbb{P}[D] \} = 0 
\end{align}

\newpage %% 2
\section{}
Call the event $W \leq 0$ to be $\mathcal{E}_{\leq}$ and $W \geq 0$ to be $\mathcal{E}_\geq$. Then, note that $\mathcal{E}_\leq \cap \mathcal{E}_\geq$ is precisely the event where $W = 0$. Thus, by provided theorems, we know that:
\begin{align}
    \mathbb{P}[\mathcal{E}_\leq \cup \mathcal{E}_\geq] &= \mathbb{P}[\mathcal{E}_\leq] + \mathbb{P}[\mathcal{E}_\geq] - \mathbb{P}[\mathcal{E}_\leq \cap \mathcal{E}_\geq]  \\
    1 &= c + d - \mathbb{P}[W = 0] \\
    \mathbb{P}[W = 0] &= c + d - 1
\end{align}


% \begin{align}
%     \mathbb{P}[\mathcal{E}_\leq \cap \mathcal{E}_\geq] \leq \min \left(\mathbb{P}[\mathcal{E}_\leq], \mathbb{P}[\mathcal{E}_\geq]\right) = \min(c, d)
% \end{align}

% Furthermore, we can produce a lower bound on the event by recognizing $\mathcal{E}_\leq \cup \mathcal{E}_\geq = \mathcal{S}$ where $S$ is the sample space, so that:

% \how{i need to lower bound lmao}

\newpage %% 3
\section{}
\subsection{}
No - because $\mathbb{P}[A \cap B] \neq 0$, $A, B$ are not mutually exclusive. 

\subsection{}
From prior theorems, note that:
\begin{align}
    \mathbb{P}[A \cap B] \leq \mathbb{P}[B] \leq \mathbb{P}[A \cup B] \implies \mathbb{P}[B] \in [1/8, 1/2]
\end{align}

\subsection{}
We will employ a proof by contradiction. Suppose that $A \indep B$, so that:
\begin{align}
    \frac{1}{8} = \mathbb{P}[A] \mathbb{P}[B]
\end{align}

Consider that:
\begin{align}
    \mathbb{P}[A \cup B] = \mathbb{P}[A] + \mathbb{P}[B] - \mathbb{P}[A \cap B] \implies \frac{5}{8} - \mathbb{P}[B]= \mathbb{P}[A]
\end{align}

This requires:
\begin{align}
    \frac{1}{8} = \left( \frac{5}{8} - \mathbb{P}[B] \right) \mathbb{P}[B] \implies \mathbb{P}[B]^2 - \frac{5}{8}\mathbb{P}[B] + \frac{1}{8} = 0
\end{align}

We can solve this via the quadratic equation, but it's unnecessary; a simple test of $b^2 - 4ac$ demonstrates that there are no real solutions for $\mathbb{P}[B]$:
\begin{align}
    \frac{25}{64} - \frac{1}{2} < 0
\end{align}

Thus, our assumption was false and we cannot have that $A \indep B$. 




\newpage %% 4
\section{}
\subsection{}
Recall that $\mathbb{P}[A \cup B] \geq \max(\mathbb{P}[A], \mathbb{P}[B])$, so $r \geq \max(\mathbb{P}[A], \mathbb{P}[B]), s \geq \max(\mathbb{P}[A], \mathbb{P}[C])$. Thus, it's conceivable that, when $\mathbb{P}[B] = r, \mathbb{P}[C] = s$, that we can allow $\mathbb{P}[A] = 0$.

On the other hand, let's consider the maximal value $\mathbb{P}[A]$ could be. Note that $\mathbb{P}[A] \leq \mathbb{P}[A \cup B], \mathbb{P}[A] \leq \mathbb{P}[A \cup C]$ by the above statements. Thus, $\mathbb{P}[A] \leq \min(\mathbb{P}[A \cup B], \mathbb{P}[A \cup C]) = \min(r, s)$. So, we conclude:
\begin{align}
    0 \leq \mathbb{P}[A] \leq \min(r, s)
\end{align}

\subsection{}
\how{um who knows lmao}

\newpage %% 5
\section{}
\subsection{}

\subsection{}

\subsection{}

\subsection{}

\subsection{}

\subsection{}

\subsection{}

\newpage %% 6 
\section{}
\subsection{}

\subsection{}

\subsection{}

\subsection{}

\newpage %% 7
\section{}
\subsection{}
\begin{proof}
We seek to show that $A^C \indep B^C$ by demonstrating that $\mathbb{P}[A^C \cap B^C] = \mathbb{P}[A^C] \mathbb{P}[B^C]$. Recall that, if we partition the sample space, it's probability must sum to 1. Thus,
\begin{align}
    \mathbb{P}[A \cup B] + \mathbb{P}[\widetilde{A \cup B}] &= 1 \\
    \mathbb{P}[A] + \mathbb{P}[B] - \mathbb{P}[A \cap B] + \mathbb{P}[A^C \cap B^C] &= 1 \\
    \mathbb{P}[A^C \cap B^C] &= 1 - \mathbb{P}[A] - \mathbb{P}[B] + \mathbb{P}[A] \mathbb{P}[B] \\
    &= 1 - \mathbb{P}[A] - \mathbb{P}[B](1 - \mathbb{P}[A]) \\
    &= (1 - \mathbb{P}[B])(1- \mathbb{P}[A]) \\
    \mathbb{P}[A^C \cap B^C] &= \mathbb{P}[A^C] \mathbb{P}[B^C] 
\end{align}

As desired.
\end{proof}


\subsection{}

\subsection{}
$A \indep B$:
\begin{proof}
We seek to show $\mathbb{P}[A \cap B] = \mathbb{P}[A] \mathbb{P}[B]$. Note that $\mathbb{P}[A]$ is $\frac{1}{6}$, as is $\mathbb{P}[B]$. We also know that $\mathbb{P}[A \cap B] = \frac{1}{36}$, because there is only one possibility where both die are 6 $(6, 6)$, while there are 36, equally likely possibilities for both die roles. Thus,
\begin{align}
    \mathbb{P}[A \cap B] = \frac{1}{36} = \frac{1}{6} \cdot \frac{1}{6} = \mathbb{P}[A] \mathbb{P}[B]
\end{align}
\end{proof}

$A \not \indep B | D$ \how{?}
\begin{proof}

\end{proof}


\printbibliography
\end{document}



% This is actually the Taylor series for $e^x$; so, substituting, we see:

% \begin{align}
%     &= 2 \Big[ \sum_{k = 0}^{\infty} \frac{1}{k!} \Big(\frac{2\lambda t}{N}\Big)^k - 1 - \Big(\frac{2\lambda t}{N}\Big) \Big]
% \end{align}